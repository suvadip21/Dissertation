% Chapter 1

\chapter{Introduction} % Main chapter title

\label{Chapter1} % For referencing the chapter elsewhere, use \ref{Chapter1} 

\lhead{Chapter 1. \emph{Intro.}} % This is for the header on each page - perhaps a shortened title

%----------------------------------------------------------------------------------------

Where the genome project mapped the genetic structure of complicated organisms such as the mouse, those pursuing the \textit{neurome} are seeking the same for the neural anatomy. In the recent years, thanks to the significant progress in biology and imaging techniques, this daunting challenge of understanding the brain appears more achievable. In fact, with the considerable amount of image data readily available using modern imaging techniques, the onus is on the signal and image processing community to contribute towards the computational aspects of the problem. In fact, informatics, not bioimaging or biology itself, remains as the major roadblock in creating a neurome for complex organisms.  

The brain’s functionalities are largely governed by its neurons, and the number of neurons vary between a few hundreds in the roundworm \textit{C. elegans}\cite{cElegans} to a hundred billion in an adult human brain. The relationship between the morphology and functionality of neurons was established by Ramon Cajal in the 19th century. Cajal’s hypothesis serves as the basis for modern day neuro-image analysis. Morphological analysis of individual neurons and neuronal components such as dendritic spines, synapses, mitochondria etc. has shown promise in better understanding and diagnosis of various neurological disorders and neuro-degenerative diseases \cite{bio_belichenko1994rett,neuron_structure,barry_serotonergic,barry_branching,cuntz_neuron}. It is evident that neuro-image analysis becomes a big data problem as we prepare ourselves to study the nervous system of developed species. This suggests that the prevalent norm of data interpretation by a trained human personnel needs to be replaced with sophisticated automation. It is not surprising that this problem has been receiving significant attention over the last few years. For example, the publicly accessible website \textit{neuromorpho.org} \cite{neuromorpho} was published in 2006 with only a few hundreds of neurons in its repository. As of June 2015, neuromorpho.org contains more than ten thousand digitally reconstructed neurons, contributed by researchers from over 120 laboratories worldwide.

\section{Neuroimage Analysis}
A system level overview of a neuro-image analysis method would consist of the following basic components – image acquisition, object detection (segmentation) and structural analysis of the detected object \cite{meijering_survey}. In the following sections, we will discuss these steps in detail.

\subsection{Image acquisition}
Choice of a particular imaging modality depends on the specific application. Fluorescence microscopy is a popular choice when the study involves a global structural analysis of the neurons or some neuronal components in the micrometer scale. For such imaging techniques, the specimen is tagged with a fluorescence protein (GFP, YFP etc.) which emits photons when illuminated by a light source \cite{barry_branching}. These photons are eventually detected by a sensor to produce an image of an optical plane. Laser scanning confocal microscopes are commonly used for fast three dimensional imaging of neurons of model animals such as Drosophila, rat, mice etc. Depending on the application, other imaging techniques such as bright-field microscopy \cite{oberlaender2007transmitted}, multiphoton microscopy \cite{santamaria2007automatic} etc. are also used to image neuronal structures.   

Electron microscopy (EM) is a popular choice for imaging neuronal structures at nanometer scale. EM is particularly useful in analyzing subcellular objects and surrounding structures such as mitochondria, synapse, vesicles etc. Focus Ion Beam Scanning Electron Microscopy (FIBSEM) \cite{kreshuk2011automated} can now deliver near isotropic 3D images with extremely high resolution and is slowly being the imaging modality of choice for nano-scale analysis of neuronal structures. 

\subsection{Image analysis}
While we are still far away from achieving our end goal of understanding the brain, recent research suggest that detection and quantification of morphological anomalies of some neuronal structures can answer some relevant questions related to diagnosis of certain neural disorders. Specifically, morphological structure of individual neurons, dendritic spines and certain characteristics of subcellular objects such as synapses, mitochondrion etc. reveal important information regarding the brain’s functioning. Anomaly quantification can be performed via comparison of the shape of the structures, which in turn requires a robust segmentation technique. Broadly, the relevant research in neuro-image analysis can be categorized into the following groups: segmentation and shape analysis of individual neurons \cite{dima_wavalet,mukherjee_T2T_2,mukherjee_TuFF,rodriguez_voxelscoop,peng_GAD}, study of the types of dendritic cells and characteristics of the intra neuronal structures\cite{5613939,6008641,6971126,EMmembrane_nguyen}. While the end goal remains the same, all these methods differ considerably from the engineering point of view and require different imaging modalities. As a result, the processing algorithms differ considerably in nature, thus making each of these techniques individual topic of extensive research.

In the recent years there have been concerted efforts to develop analytic models for global morphological comparisons of neurons. This is because anatomical distortion of neurons provide initial clues toward neurological disease understanding, diagnosis or  monitoring. Global structure analysis of neurons require a two stage pipeline. First, a digital reconstruction should be obtained from the raw image data. This is the segmentation or tracing stage. With the reconstruction available, the next challenge is to devise a method to compare the structures mathematically. It turns out that both these sub-problems come with their own sets of challenges and complications and deserve to be treated separately. 

The pertinent challenge for global structural analysis is to develop appropriate pipeline for identification and quantification of the morphology of a single neuron. Confocal microscopy is generally the chosen modality for imaging the neuronal structures or \textit{neurites}, since the structures are visible in the micrometer resolution. Neuron reconstruction (or tracing) refers to the problem of acquiring the neural geometry from raw microscopy image.  Image processing is challenging both due to the structural complexity of neurons as well as due to imaging artifacts such as poor contrast, presence of non-neuronal clutter and low signal to noise ratio of the images. The objective is to perform 3D segmentation, which requires proper care to handle the filament bifurcations as well as deal with the sporadic signal attenuation due to inhomogeneous staining of the specimen with fluorescent dye. 

%\subsection{Neuron reconstruction posed as a vessel detection problem}
%
%As mentioned earlier, neurites are tubular structures which can be appropriately modeled as  tree shaped objects with varying degrees of branch bifurcations that determines its structural complexity. Several neuron reconstruction algorithms draw inspiration from the works in medical image analysis which focus on segmenting tubular or vascular objects from images. There have been a few methods proposed in the literature in this regard for different applications and modalities, such as segmentation of retinal blood vessels, human arteries from Computed Tomography Angiography images for detecting aneurysms \cite{gooya2012generalization,gooya2008variational,lesage2009review,shang2011vascular,nain2004vessel,jacob2004steerable,manniesing2006vessel,sofka2006retinal} etc.  Other non biomedical applications of vessel detection in the computer vision community include detection of tubular structures (such as roads, bridges etc.) from aerial images \cite{gonzalez_2010,turetken_MIP}, identifying cracks on concrete structures such as pavements and bridges \cite{oliveira2013automatic,crackd_TASE,oliveira2014crackit} etc. 
%
%However, despite the fact that the problem of vessel detection has been studied for quite some time, direct adaptation of an off-the-shelf algorithm to a particular task is still non trivial. This is due to the fact that each imaging application, with its associated modality pose different challenges in terms of denoising, object enhancement and clutter removal. For example, fluorescence microscopy images are often degraded by photon noise, inhomogeneous brightness of the objects and sporadic signal attenuation, which makes segmentation difficult. This requires an application specific approach that respects the local morphology of the structures, but is robust to the various imaging artifacts as well.  

\section{Scope of the dissertation}
The major emphasis of this thesis will be on developing novel  algorithms for the purpose of segmenting single neurons from confocal microscopy data. We realize that a large scale structural analysis of neuron groups demand efficient, automated segmentation to generate the digital morphology. Therefore, in this work, we primarily focus on developing and improving the first stepping stone for \textit{neuromics}-- automated neuron segmentation algorithms. 

We start with a 2-d framework, and gradually progress to the more complicated 3-d segmentation problem. We identify the key issues which are necessary for robust neuron structure detection viz. prior enhancement of tubular neurites and the ability to deal with abrupt signal attenuation due to imaging artifacts. The segmentation algorithms are formulated so as to adequately respond to these issues. Finally, we propose a modification and improvement for the neuron enhancement step, which is an integral aspect for both the segmentation algorithms. We further show that the developed and proposed methodologies can also be used for a wide variety of applications which scale from bio-imaging to civil engineering. The specific aims of this dissertation are given below:
%
%\textit{The overall arrangement of the dissertation is as follows:
%
%\begin{itemize}
%\item \textbf{Chapter 1}: Introduction to the problem
%\item \textbf{Chapter 2}: Motivation and detailed background survey of segmentation of vascular structures, with an emphasis on neuron segmentation from confocal microscopy.
%\item \textbf{Chapter 3}: Motivation for using geometric active contours for seggmentation. Discussion of relevant level set based methodologies.
%\item \textbf{Chapter 4}: In this chapter we focus on neuron segmentation from 2D images and we propose a solution which tackles the inhomogeneity in the object illumination. 
%\item \textbf{Chapter 5}: In this chapter, we discuss our 3D neuron segmentation algorithm "Tubularity Flow Field (TuFF)". Qualitative and quantitative evaluation on a set of 3D confocal images  are presented. 
%\item \textbf{Chapter 6}: In this chapter, we propose a robust version of TuFF. This model incorporates a more suitable pre-processing step along with a component to reduce segmentation error due to contour leakage. A non biological application is discussed, which involves automated identification of cracks from concrete structures. 
%\item \textbf{Chapter 7}: Finally, in this chapter we conclude this thesis by discussing its salient features and identifying the future prospects. 
%\end{itemize}
%}

\subsubsection*{Aim 1: Algorithm for neuron segmentation in 2D}

Our first goal is to develop a generalized segmentation policy for  2D microscopy images. 2D analysis often serves as a preliminary step for understanding the anatomy of neurites. Furthermore, certain categories of neurons (e.g. sensory neurons on the cuticle layer of \textit{Drosophila} larva) are topologically flat and therefore, the third dimension of imaging does not yield useful information for analysis. 

In this regard, we have developed a general purpose segmentation algorithm which uses geometric active contours. The proposed algorithm, \textit{Legendre Level Set} (L2S) aims at segmenting the objects from microscopy images in presence of heterogeneous illumination. We also briefly discuss a closely related algorithm \textit{Dictionary Learning Level Set}, which uses dictionary learning to tackle the intensity inhomogeneity. 

\subsubsection*{Aim 2: Algorithm for neuron segmentation in 3D}
Morphological characteristics of a vast majority of neurons are better captured using 3D imaging. A vast majority 3D images of the Green Fluorescence Protein (GFP) stained adult \textit{Drosophila} fruit fly is obtained at the Dept. of Biology, University of Virginia using confocal microscope. The GFP is expressed in single neurons using a heat shot activated scheme, which allows imaging at the single cell resolution\cite{barry_branching}.

First, we discuss a graph theoretic segmentation algorithm \textit{Tree2Tree-2}, where the idea is to treat the neuron connectivity analysis problem using graph based algorithms. Details of Tree2Tree-2 are presented in Chapter 2. The second method, \textit{Tubularity Flow Field} or TuFF uses geometric active contours to perform segmentation guided by the local tubularity of the neurites. One advantage of using geometric active contours is that these techniques are adaptive to the topology of the objects. As a result, joining disjoint neurites can be handled in a natural framework unlike Tree2Tree-2, where error is often introduced in the solution due to improper connectivity handling. We also provide a mechanism to combat the sporadic signal loss across the structures by incorporating a specialized attraction force in our solution to merge nearby fragments. 

\subsubsection*{Aim 3: Robust TuFF and other applications}

Tubularity Flow Field is an effective mode of segmenting vascular shapes both in 2D and 3D applications. However, there are some aspects of TuFF which limits its broad applicability. To address these needs, we discuss a robust algorithm known as \textit{Edge Assisted TuFF} or EATuFF. The major highlights of EATuFF over TuFF is the addition of a specialized preprocessing step called Local Directional Evidence (LDE) which uses non local steerable filters to identify tubular structures in low contrast imagery. Second, an edge based attraction term is associated with the TuFF segmentation framework to reduce the contour leakage phenomenon which may occur if the image contrast is low. 

We further demonstrate the applicability of our algorithm for a non biological application which involves detecting cracks on concrete structures. Although the application differs significantly from microscopy, these cracks can also be modeled as tubular objects. We show that EATuFF can be efficiently applied for this application with promising results.

\subsection{Organization of the dissertation}
The rest of the dissertation is organized as follows: In Chapter 2, we discuss the  popular strategies for segmenting and enhancing vascular structures. We focus our attention to the problem of detecting neurites from noisy images and discuss two graph-based methodologies-- Tree2Tree and its successor Tree2Tree-2. We analyze the pros and cons of these algorithms and discuss the motivation for using geometric active contours for this problem.

In Chapter 3, a brief summary of geometric active contours is presented, followed by discussion of the proposed 2D algorithms L2S and DL2S in Chapter 4. The 3D segmentation case using TuFF is presented in Chapter 5 and in Chapter 6 we modify TuFF and show the applicability of our technique in detecting cracks from images of concrete structures. Finally, we conclude in Chapter 7 with discussion of the methods, their future extensions and possible applications.