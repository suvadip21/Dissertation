% Chapter 1

\chapter{Introduction} % Main chapter title

\label{Chapter1} % For referencing the chapter elsewhere, use \ref{Chapter1} 

\lhead{Chapter 1. \emph{Intro.}} % This is for the header on each page - perhaps a shortened title

%----------------------------------------------------------------------------------------

Where the genome project mapped the genetic structure of complicated organisms such as the mouse, those pursuing the \textit{neurome} are seeking the same for the neural anatomy. In the recent years, thanks to the significant progress in biology and imaging techniques, this daunting challenge of understanding the brain appears more achievable. In fact, with the considerable amount of image data readily available using modern imaging techniques, the onus is on the signal and image processing community to contribute towards the computational aspects of the problem. In fact, informatics, not bioimaging or biology itself, remains as the major roadblock in creating a neurome for complex organisms.  

The brain’s functionalities are largely governed by its neurons, and the number of neurons vary between a few hundreds in the roundworm \textit{C. elegans}\cite{cElegans} to a hundred billion in an adult human brain. The relationship between the morphology and functionality of neurons was established by Ramon Cajal in the 19th century. Cajal’s hypothesis serves as the basis for modern day neuro-image analysis. Morphological analysis of individual neurons and neuronal components such as dendritic spines, synapses, mitochondria etc. has shown promise in better understanding and diagnosis of various neurological disorders and neuro-degenerative diseases \cite{bio_belichenko1994rett,neuron_structure,barry_serotonergic,barry_branching,cuntz_neuron}. It is evident that neuro-image analysis becomes a big data problem as we prepare ourselves to study the nervous system of developed species. This suggests that the prevalent norm of data interpretation by a trained human personnel needs to be replaced with sophisticated automation. It is not surprising that this problem has been receiving significant attention over the last few years. For example, the publicly accessible website \textit{neuromorpho.org} \cite{neuromorpho} was published in 2006 with only a few hundreds of neurons in its repository. As of June 2015, neuromorpho.org contains more than ten thousand digitally reconstructed neurons, contributed by researchers from over 120 laboratories worldwide.

\section{Neuroimage Analysis}
A system level overview of a neuro-image analysis method would consist of following basic components – image acquisition, object detection (segmentation) and structural analysis of the detected object \cite{meijering_survey}. In the following sections, we will discuss these steps in detail.

\subsection{Image acquisition}
Choice of a particular imaging modality depends on the specific application. Fluorescence microscopy is a popular choice when the study involves a global structural analysis of the neurons or some neuronal components in the micrometer scale. For such imaging techniques, the specimen is tagged with a fluorescence protein (GFP, YFP etc.) which emits photons when illuminated by a light source \cite{barry_branching}. These photons are eventually detected by a sensor to produce an image of an optical plane. Laser scanning confocal microscopes are commonly used for fast three dimensional imaging of neurons of model animals such as Drosophila, rat, mice etc. Depending on the application, other imaging techniques such as bright-field microscopy \cite{oberlaender2007transmitted}, multiphoton microscopy \cite{santamaria2007automatic} etc. are also used to image neuronal structures.   

Electron microscopy (EM) is a popular choice for imaging neuronal structures at nanometer scale. EM is particularly useful in analyzing subcellular objects and surrounding structures such as mitochondria, synapse, vesicles etc. Focus Ion Beam Scanning Electron Microscopy (FIBSEM) \cite{kreshuk2011automated} can now deliver near isotropic 3D images with extremely high resolution and is slowly being the imaging modality of choice for nano-scale analysis of neuronal structures. 

\subsection{Image analysis}
While we are still far away from achieving our end goal of understanding the brain, recent research suggest that detection and quantification of morphological anomalies of some neuronal structures can answer some relevant questions related to diagnosis of certain neural disorders. Specifically, morphological structure of individual neurons, dendritic spines and certain characteristics of subcellular objects such as synapses, mitochondrion etc. reveal important information regarding the brain’s functioning. Anomaly quantification can be performed via comparison of the shape of the structures, which in turn requires a robust segmentation technique. Broadly, the relevant research in neuro-image analysis can be categorized into the following groups: segmentation and shape analysis of individual neurons \cite{dima_wavalet,mukherjee_T2T_2,mukherjee_TuFF,rodriguez_voxelscoop,peng_GAD}, study of the types of dendritic cells and characteristics of the intra neuronal structures\cite{5613939,6008641,6971126,EMmembrane_nguyen}. While the end goal remains the same, all these methods differ considerably from the engineering point of view and require different imaging modalities. As a result, the processing algorithms differ considerably in nature, thus making each of these techniques individual topic of extensive research.

In the recent years there have been concerted efforts to develop analytic models for global morphological comparisons of neurons. This is because anatomical distortion of neurons provide initial clues toward neurological disease understanding, diagnosis or  monitoring. Global structure analysis of neurons require a two stage pipeline. First, a digital reconstruction should be obtained from the raw image data. This is the segmentation or tracing stage. With the reconstruction available, the next challenge is to devise a method to compare the structures mathematically. It turns out that both these sub-problems come with their own sets of challenges and complications and deserve to be treated separately. 

The pertinent challenge for global structural analysis is to develop appropriate pipeline for identification and quantification of the morphology of a single neuron. Confocal microscopy is generally the chosen modality for imaging the neuronal structures or \textit{neurites}, since the structures are visible in the micrometer resolution. Neuron segmentation or neuron tracing refers to the problem of acquiring the neural geometry from raw microscopy image.  Image processing is challenging both due to the structural complexity of neurons as well as due to imaging artifacts such as poor contrast, presence of non-neuronal clutter and low signal to noise ratio of the images. The objective is to perform 3D segmentation of the neuron, which requires proper care to handle the filament bifurcations as well as deal with the sporadic signal attenuation due to inhomogeneous staining of the specimen with fluorescent dye. 

\subsection{Algorithms for detecting vascularity}

\section{Scope of the dissertation}
The major emphasis of this thesis will be on developing novel  algorithms for the purpose of segmenting neurons from confocal microscopy data. The final result of the segmentation is the neuronal morphology, embedded in a graph theoretic tree for further shape based analysis. We realize that a large scale structural analysis of neuron groups demand efficient, automated segmentation to generate the digital morphology. Therefore, in this work, we primarily focus on developing and improving the first stepping stone for \textit{neuromics}-- automated neuron segmentation algorithms. We start with a 2-d framework, and gradually progress to the more complicated 3-d segmentation problem. As will be discussed in this proposal, we identify the key issues which are necessary for robust neuron structure detection viz. prior enhancement of tubular neurites and the ability to deal with abrupt signal attenuation due to imaging artifacts. The segmentation algorithms are formulated so as to adequately respond to these issues. Finally, we propose a modification and improvement for the neuron enhancement step, which is an integral aspect for both the 3-d segmentation algorithms. We further show that the developed and proposed methodologies can also be used for a wide variety of applications which scale from bio-imaging to civil engineering. 
