% Chapter 2

\chapter{Background} % Main chapter title

\label{Chapter2} % For referencing the chapter elsewhere, use \ref{Chapter2} 

\lhead{Chapter 2. \emph{Background}} % This is for the header on each page - perhaps a shortened title

%----------------------------------------------------------------------------------------

In this chapter we review some relevant research in neuron segmentation. Since the primary objective of this dissertation is to segment neurons from confocal microscopy images, techniques which use other imaging modalities (such as electron microscopy) are excluded from this discussion.

We can broadly categorize the neuron segmentation schemes in two basic approaches. The first set of methods use user defined (or automatically detected) initial seed points to perform tracing. The second category of algorithms avoid seed initialization and perform segmentation globally.

Manual seed selection has the advantage that the segmentation region is identified a priory by an expert. This introduces locality in processing, which results in higher processing speed. Typically such algorithms generate the neuronal tree from semi-automatically initialized seed points on the neurite centerlines. Al-Kofahi \textit{et al}. \cite{al_kofahi} used the medial response of multiple directional templates to determine the  direction to generate successive seed points along the neuron medial axis. This local tracing method shows good performance in high-contrast images, but requires continuity in the neuron branches for reliable segmentation. 

Segmentation performance can be considerably improved if the seed points are selected manually. These seeds are then treated as nodes in a graph, and segmentation is performed using graph theoretic algorithms. When seed selection is done automatically, a pruning step is generally used to eliminate the non-neuronal points. With this optimal set of seeds, the methods in \cite{peng_anisotropicPS,peng_GAD,peng_APP} establish connectivity between the nodes using a shortest path algorithm \cite{dijkstra1959note}, by suitably selecting the weights on the graph edges. Fast and accurate segmentation is possible using the above mentioned approaches if the neuron structure is morphologically simple and the image noise level is low. Gonzalez \textit{et al}. \cite{gonzalez_2010} introduced a graph theoretic technique to delineate the optimal neuronal tree from an initial set of seeds by computing a K-Minimum Spanning Tree. An approximate solution to this NP-hard problem was realized by minimizing a global energy function in a linear integer programming framework. However, due to its  greedy nature, the algorithm may converge to undesired local minima. 

We hypothesize that seed based techniques are useful if the imaged neurons are not too complicated structurally. In such scenarios, where manual seed selection is easy, reliable segmentation can be obtained. However, since automatically choosing the correct set of seed points is still an open problem, it is difficult to use the above mentioned techniques for high throughput, no intervention analysis. Also, since proper selection of seeds points is instrumental in these methods, the segmentation accuracy is sometimes compromised if a sub-optimal set of points is chosen. Furthermore, the connectivity analysis between the seeds assume uniform signal intensity, and noise and low contrast in the images may degrade the segmentation quality.

In contrast to the seed based local techniques, traditional segmentation approaches are more global, typically requiring an initial pre-processing of the image followed by a specialized segmentation step. Although a global approach may suffer from expensive computation, they are more suitable for neurite junction and end point detection.
Typically, such methods rely on a four stage processing pipeline -- enhancement, segmentation, centerline detection and post processing. The voxel scooping algorithm proposed in \cite{rodriguez_voxelscoop} assumes tubular structure of the neurite filaments and iteratively searches for voxel clusters in a manner similar to region growing. A pruning step is then deployed to eliminate spurious end nodes. A similar region growing method is implemented in the popular automatic neuron tracing tool Neuronstudio\cite{wearne_neuronStudio}. The segmentation step is generally followed by a centerline detection \cite{cuntz_neuron,mukherjee_medialness} stage to detect the medial axis of the segmented structure. In many cases further smoothing of the medial axis is performed by spline fitting \cite{basu_T2T_journal}. Since such methods do not rely on human intervention, it is evident that the segmentation quality would depend heavily on the initial segmentation, which may be affected by the noise and clutter in the images.

Tree2Tree \cite{basu_T2T_journal} and its variants \cite{mukherjee_T2T_2} propose to solve the neuron segmentation problem in a graph theoretic framework. However, unlike traditional seed selection approaches, where manually initialized points are treated as the nodes of the graph, an initial segmentation algorithm is devised to produce disjoint connected components. Connectivity between the components is analyzed based on their separating distance and orientation, which determines the weights of the graph edges to perform segmentation using a minimum spanning tree approach. 

Although the primary contribution of Tree2Tree is to connect the fragmented neurite segments automatically, this connectivity analysis relies on heavily on the initialization. Noise and clutter in the images create undesired artifacts in the global segmentation, resulting in loss of structural information. Moreover, linking the components based on their relative geometric orientation requires computation of the leaf-tangents from the object centerlines, which is sensitive to the irregularities of the neurite surface. Furthermore, elimination of false nodes from the neuronal tree is  difficult, and ultimately requires further manual parameter tuning.

Segmentation based on active contours \cite{kwt_snakes} have also been proposed \cite{wang_Roysam_open_curve}, \cite{cai_ISBI} to directly obtain the neuron centerline, without performing a global thresholding. The algorithm proposed by Wang \textit{et al}. \cite{wang_Roysam_open_curve}  involves evolution of an open ended snake guided by a force field that encourages the neuron trace to lie along the filament centerline. A pre-processing step based on tensor voting \cite{roysam_tensorvoting} was introduced to enhance the vascular structure of the neurites. Combined with a post-processing step to eliminate false filaments, this method is efficient in segmenting neuronal structures from low SNR confocal stacks. However, due to the inability of parametric active contours to naturally handle topological changes such as object merging, neurite branch point detection depends requires a non-trivial post processing to determine snake merging at the junctions. 

Santamaria-Pang \textit{et al}. \cite{santamaria2007automatic} use a multistage procedure for detection of tubular structures in multi-photon imagery, which includes a pre-filtering stage to identify the filaments based on supervised learning. This requires offline learning of the model parameters and prior knowledge about the vessel appearance information, which necessitates a set of accurate training examples and demands extensive human involvement to generate the ground truth. Zhou \textit{et al}. \cite{zhao_variational} propose a variational framework based on geodesic active contours to identify neurite branches from two photon microscopy. This strategy is effective when the edge information is reliable, and hence depends on efficient pre-processing to eliminate image irregularities. However, both these methods do not deploy additional schemes to identify and analyze the broken neurite fragments in their model, and hence it demands a specialized post-processing step. 
